\chapter{Conclusion, Future Projects}

The process of designing a power amplifier through the iterative process of simulation, fabrication, and measurement was a great learning experience.  The final version of the amplifier was able to meet the specifications required from the competition. For students that are looking to compete in the future it is recommended that they explore the Doherty amplifier configuration. The Doherty amplifier was not discussed in this thesis but was explorered during the literature search. The Doherty amplifier is able to maintain high PAE and drain efficiency over a wide range of input powers and with lower inter-modulation distortion than a class F amplifier by using two amplifiers with different bias points. The author decided not to pursue to the Doherty amplifier configuration due to the complexity required and time constraints. The past competition winners that have used the Doherty amplifier have always been a group endeavour with a group member assigned to each amplifier then another to combining the amplifiers in the Doherty configuration. The class F amplifier seems to be better suited for applications with a fixed transmission power because then the high efficiency will always be maintained. Also the class F amplifier seems better for systems that use constant amplitude modulation schemes like CW, FM, or FSK due to high degree of non-linearity which make's it challenging to use for modulation schemes that use phase to transmit information.

Another aspect not discussed in this thesis is active gate biasing. Wolfspeed, the manufacturer of the transistor used, recommends changing the gate voltage 0.4 mV per degree Celsius in order to maintain a constant drain current. The efficiency of the amplifier would most likely be increased if this was done but due to time constraints only a constant bias voltage was used during testing. Wolfspeed has an application note that has an example circuit for an active gate bias power supply. Digital predistortion could also be investigated to reduce the intermodulation distortion (IMD) of the class F amplifier. Digital predistortion is not allowed for the IMS design competition but would be an interesting avenue to investigate to improve the linearity of the class F amplifier.
