\chapter{Abstract and IEEE Design Competition Rules}
Telecommunications is the basis of our modern world. There is infrastructure from microwave links, fiber-optic links, and satellite links that connect people all over the world together. Each year more and more data is transferred through these networks. The Cellular Telephone Industries Association reported less than 100 MB of average monthly traffic per smartphone in the United States in 2009 and in 2014 this has jumped up to over 1800 MB of average monthly traffic. To transfer more data, higher order modulation schemes must be used like QAM which require linear amplifiers so amplitude and phase is not distorted during amplification. In portable devices were battery life is one of the key parameters, high efficiency power amplifiers are important due to the large power consumption of the RF amplifiers.

 The International Microwave Symposium (IMS) 2016 High Efficiency Power Amplifier Student Design competition focuses on maximizing the power added efficiency (PAE) and operating frequency of the amplifier over a wide range of input powers. Power added efficiency is defined as the difference of output RF power and RF input power over the DC bias power seen in Equation \ref{eq:pae}. PAE is an important metric for power amplifiers to show how efficient the amplifier is converting the DC power into RF power, ideally converting 100\%.

\begin{equation}\label{eq:pae}
  PAE,\% = \frac{RF\;Power_{out} - RF\\Power_{in}}{DC\;Power}
\end{equation}

To qualify for the design competition the amplifier must output at least 36 dBm (4 Watts) with less than 24 dBm input power for a single carrier and less than 22 dBm per tone for two carriers spaced 5 MHz apart. Also the amplifier must have a carrier-to-intermodulation ratio (C/I), defined as the ratio in dB between the amplitude of either carrier and the highest intermodulation product, greater than 30 dB at an input power of 0 dBm.

The rules for the amplifier require the amplifier have a center frequency between 1 GHz and 10 GHz. The RF ports should use SMA female connectors, and the bias connections should use banana plugs with a maximum of two DC supply voltages. The source and load impedance of your amplifier should be matched to 50 $\Omega$. The amplifier can use any technology but must be the result of new design and fabrication methods. The amplifier PAE will be measured with at the first RF output power when the C/I ratio drops below 30 dB. The winning amplifier will have the highest figure of merit seen in Equation \ref{eq:fom}. 

\begin{equation}\label{eq:fom}
  Figure of Merit = PAE*(Center Frequency, GHz)^{0.25}
\end{equation}

A table of past competition winners can be seen in Table \ref{table:past_amp_val}. The highest scores in the competition have all been from class F amplifiers with a center frequency around 3.5 GHz with the best amplifier in the competitions past scoring 109.2 at a center frequency of 3.475 GHz and a PAE of 80.1\% in 2011. Since 2011 the winning figure of merit has dropped by more than 20 points. The author thinks that in 2011 and earlier the amplifiers were scored at their highest PAE. And from 2012 on, the amplifiers were scored at the input power when the C/I ratio drops below 30 dB. The class F amplifier was chosen for the design due to highest PAE compared to other amplifier classes and because the design could be accomplished solo within an academic year. Most of the submissions were a group effort, especially the Doherty amplifiers which use a combination of two discrete amplifiers in the design. As per the rules, the amplifier has to be a result of a new effort and this work explores coupling the stubs of the harmonic matching networks to reduce the electrical size of the amplifier using a gallium nitride (GaN) high electron mobility (HEMT) transistor.

\begin{table}
    \centering

    \label{table:past_amp_val}
    \caption{Table of Past Competition Winners and Amplifier Parameters}

    \begin{tabular}{|l|l|l|l|l|l|}
      \hline
      % after \\: \hline or \cline{col1-col2} \cline{col3-col4} ...
      {Year} & {Device} & {Class} & {Frequency, GHz} & {PAE, \%} & {Figure of Merit} \\ \hline
%      2005 & {GaAs FET} & {N/A}     & {1.5}   & {61.7} & {68.3} \\ \hline
      2006 & {Si LDMOS} & {$F^{-1}$}  & {1}     & {75.9} & {75.9} \\ \hline
      2007 & {GaN HEMT} & F         & {1.21}  & {82.9} & {87.0} \\ \hline
      2008 & {GaN HEMT} & {$F^{-1}$}  & {3.2}   & {71.9} & {96.1} \\ \hline
      2009 & {GaN HEMT} & {$F^{-1}$}  & {3.27}  & {71.1} & {95.6} \\ \hline
      2010 & {GaN HEMT} & {J}       & {3.5}   & {74.7} & {102.1} \\ \hline
      2011 & {GaN HEMT} & F         & {3.475} & {80.1} & {109.2} \\ \hline
      2012 & {GaN HEMT} & {Doherty} & {3.5}   & {59.0} & {80.7} \\ \hline
      2013 & {GaN HEMT} & {Doherty} & {3.5}   & {62.5} & {85.5} \\ \hline
      2014 & {GaN HEMT} & {Doherty} & {5}     & {57.8} & {86.4} \\ \hline
      \hline
    \end{tabular}
\end{table}
