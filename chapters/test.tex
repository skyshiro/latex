\chapter{Class F Amplifier Test Results, Comparisons to Specs}

Talk about build process?

The final version of the amplifier was milled using INSERT MILLING MACHINE PART NUMBER. Vias were added underneath the transistor to improve the heat conduction after the board layout was done in ADS. In addition a fan was mounted on the board with a heatsink on the transistor for additional cooling. The fan draws 30 mA at 40 volts and was tied to the drain voltage of the amplifier. Two previous versions were built but due to the challenge of soldering the leadless package of the transistor they were damaged most likely due to shorts and insufficient cooling. The bias inductors, coupling capacitors, and transistor was mounted on the board using solder paste and a heatgun. A bill of materials and the board layout can be found in the Appendix. The final build of the amplifier is show in Figure X. The amplifier was measured with an Anritsu PART NUMBER HERE and the drain powered with an Agilent Power Supply PART NUMBER with the drain current being measured by an Agilent Multimeter PART NUMBER. The gate voltage was powered by an Agilent Power Supply PART NUMBER. The VNA was calibrated from x GHz to Y GHz and set to average ten samples. A 20 dB attenuator was put at the output of the amplifier so the VNA would not be damaged by the high RF power. The 20 dB attenuator was measured so the attenuation could be removed and the amplifier gain could be properly measured.

The amplifier s-parameters and drain current was measured over a wide range of input powers from -10 dBm to 20 dBm.

Need a test black box diagram for and equipment part numbers

Unable to measure high power two tone due to lack of equipment. Only able to measure IMD at low input powers.

Need photo of amplifier lol 

VSWR plots

Gain vs frequency and input power

PAE and Gain vs input power

power out vs power in
